\documentclass[a4paper,10pt]{scrartcl}
\usepackage[utf8]{inputenc}

%opening
\title{Workshop on \\ Foundations for the \\ Practical Formalization of Mathematics}
\author{}

\begin{document}
\date{Nantes, 26-27 April 2017}
\maketitle

% \begin{abstract}
% 
% \end{abstract}

\section*{Schedule}

\subsection*{Wednesday 26 April}

% \begin{center}
\begin{tabular}{ll}

10:30 - 11:15 & Tea \\
11:20 - 12:10 & Matthes \\
12:10 - 13:00  & Maggesi \\
13:00 - 15:00  & Lunch break\\
15:00 - 15:50  & Dybjer\\
15:50 - 16:40  & van der Weide \\
16:40 - 17:10  & Tea \\
17:10 - 18:00  & Luo \\
20:00 -        & Dinner
\end{tabular}
% \end{center}

\subsection*{Thursday 27 April}

% \begin{center}
\begin{tabular}{ll}

09:50 - 10:40 & Coquand \\
10:40 - 11:15  & Tea \\
11:20 - 12:10  &Wellen \\
12:10 - 13:00 & Larchey-Wendling \\
13:00 - 15:00 & Lunch break \\
15:00 - 15:50 & Adams \\
15:50 - 16:40 & Escardó \\
16:40 - 17:10 & Tea \\
17:10 - 18:00 & Kaposi \\ 
\end{tabular}
% \end{center}
  
\section*{Abstracts}

\subsection*{Adams: Why Is It Still Hard to Formalize Metatheory?}

Formalizing metatheory has been a problem since before the invention of the computer---it was the original motivation for the lambda-calculus.  
And yet, after so much work, we still do not have agreement on the best way to do it, or desirable characteristics that a formalization should satisfy.

Almost all approaches require the syntax and rules of deduction of each system to be given as inductive types.  
This makes it hard to re-use results proved about one system in another.

After surveying several approaches, I will present recent work on a library in Agda of results about formal systems, called MetaL (Metatheory Library).  
The library MetaL contains general proofs of results Weakening and Substitution Lemmas, and the Church-Rosser theorem for reduction relations that have no critical pairs.  
It was designed with these criteria in mind: 
\begin{enumerate}
 \item The definition of a syntax should look like the definition on paper.  
 \item After the syntax is defined, the general results should be immediately available for that syntax.  
 \item It should be possible to define functions by induction on syntax, and prove results by induction on syntax and induction on derivations, using Agda's built-in pattern matching.  
\end{enumerate}
 The library MetaL uses de Bruijn indices, but the technique could easily be adapted to produce a similar library for another representation.

\subsection*{Coquand: Stack model of type theory}

Since we have models of univalence in a constructive setting one would expect to use known techniques such as sheaf models to prove independence results, 
such as the fact that countable choice is independent from univalence. 
When trying this for the groupoid model (thus working with groupoids in a sheaf model) one surprise is that countable choice, formulated with propositional truncation, always hold. 
When analysing this problem, one sees that the sheaf condition is too strong: it requires the local data to be compatible w.r.t. strict equalities, 
while what is needed there should be compatibility for path equalities. This is exactly the notion of stacks, as defined in algebraic geometry. 
We will explain then the groupoid model counter-model for an example suggested by Martín Escardó and Andrew Swan, namely

\[ \left(\prod_{n:N} || B + A(n) ||\right)  \to || \prod_{n:N} B + A(n) || \]

If time permits, we will present the generalisation to cubical stacks, giving the independence of this principle also from a hierarchy of univalent universes.


\subsection*{Dybjer: Finitary Higher Inductive Types in the Groupoid Model}

A higher inductive type of level 1 (a 1-hit) has constructors for points and paths only, whereas a higher inductive type of level 2 (a 2-hit) has constructors for surfaces too. 
We restrict attention to finitary higher inductive types and present general schemata for the types of their point, path, and surface constructors. 
We also derive the elimination and equality rules from the types of constructors for 1-hits and 2-hits. 
Moreover, we construct a groupoid model for dependent type theory with 2-hits and point out that we obtain a setoid model for dependent type theory with 1-hits by truncating the groupoid model.
This is joint work with Hugo Moeneclaey, ENS Cachan



\end{document}
